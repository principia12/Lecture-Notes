\documentclass[twoside]{article}
\usepackage{import}
\usepackage{kotex}
\usepackage{lipsum} % Package to generate dummy text throughout this template

\usepackage[sc]{mathpazo} % Use the Palatino font
\usepackage[T1]{fontenc} % Use 8-bit encoding that has 256 glyphs
\linespread{1.05} % Line spacing - Palatino needs more space between lines
\usepackage{microtype} % Slightly tweak font spacing for aesthetics

\usepackage[hmarginratio=1:1,top=32mm,columnsep=20pt]{geometry} % Document margins
\usepackage{multicol} % Used for the two-column layout of the document
\usepackage[hang, small,labelfont=bf,up,textfont=it,up]{caption} % Custom captions under/above floats in tables or figures
\usepackage{booktabs} % Horizontal rules in tables
\usepackage{float} % Required for tables and figures in the multi-column environment - they need to be placed in specific locations with the [H] (e.g. \begin{table}[H])
\usepackage{hyperref} % For hyperlinks in the PDF

\usepackage{lettrine} % The lettrine is the first enlarged letter at the beginning of the text
\usepackage{paralist} % Used for the compactitem environment which makes bullet points with less space between them

\usepackage{braket}
\usepackage{array}
\usepackage{calc}
\usepackage{graphicx}
\usepackage{listings}
\usepackage{kotex}


\lstset{frame=tb,
  language=lisp,
  aboveskip=3mm,
  belowskip=3mm,
  showstringspaces=false,
  columns=flexible,
  basicstyle={\small\ttfamily},
  numbers=none,
  numberstyle=\tiny\color{gray},
  keywordstyle=\color{blue},
  commentstyle=\color{dkgreen},
  stringstyle=\color{mauve},
  breaklines=true,
  breakatwhitespace=true
  tabsize=3}

\lstnewenvironment{Python}
  {\lstset{
  language=Python, 
}}
  {}

\lstnewenvironment{C}
  {\lstset{
  language=C, 
}}
  {}
  
\lstnewenvironment{Java}
  {\lstset{
  language=Java, 
}}
  {}
  
\lstnewenvironment{scheme}
  {\lstset{
  morekeywords={define-type, define, type-case, match}
}}
  {}
  
\lstnewenvironment{expr}
  {\lstset{
	morekeywords={with, deffun, fun}
}}
  {}

\usepackage{color}
\usepackage[table,xcdraw]{xcolor}
\usepackage{adjustbox}


\definecolor{dkgreen}{rgb}{0,0.6,0}
\definecolor{gray}{rgb}{0.5,0.5,0.5}
\definecolor{mauve}{rgb}{0.58,0,0.82}



\hypersetup{%
    pdfborder = {0 0 0}
}



\usepackage{abstract} % Allows abstract customization
\renewcommand{\abstractnamefont}{\normalfont\bfseries} % Set the "Abstract" text to bold
\renewcommand{\abstracttextfont}{\normalfont\small\itshape} % Set the abstract itself to small italic text

\usepackage{titlesec} % Allows customization of titles
%\renewcommand\thesection{\Roman{section}} % Roman numerals for the sections
\renewcommand\thesubsection{\Roman{subsection}} % Roman numerals for subsections
\titleformat{\section}[block]{\large\scshape\centering}{\thesection.}{1em}{} % Change the look of the section titles
\titleformat{\subsection}[block]{\large}{\thesubsection.}{1em}{} % Change the look of the section titles

\usepackage{fancyhdr} % Headers and footers
\pagestyle{fancy} % All pages have headers and footers
\fancyhead{} % Blank out the default header
\fancyfoot{} % Blank out the default footer
\fancyhead[C]{ Deep Learning Specialization} % Custom header text
\fancyfoot[RO,LE]{\thepage} % Custom footer text

\begin{document}

%----------------------------------------------------------------------------------------
%	TITLE SECTION
%----------------------------------------------------------------------------------------

\begin{titlepage}

\newcommand{\HRule}{\rule{\linewidth}{0.5mm}} % Defines a new command for the horizontal lines, change thickness here

\center % Center everything on the page
 
%----------------------------------------------------------------------------------------
%	HEADING SECTIONS
%----------------------------------------------------------------------------------------

\vspace*{3cm}
\textsc{\Large Coursera}\\[0.5cm] % Major heading such as course name
\textsc{\large deeplearning.ai}\\[0.5cm] % Minor heading such as course title

%----------------------------------------------------------------------------------------
%	TITLE SECTION
%----------------------------------------------------------------------------------------

\HRule \\[0.4cm]
{ \huge \bfseries Deep learning specialization}\\[0.4cm] % Title of your document
\HRule \\[1.5cm]
 
%----------------------------------------------------------------------------------------
%	AUTHOR SECTION
%----------------------------------------------------------------------------------------

\begin{minipage}{0.4\textwidth}
\begin{flushleft} \large
\emph{Author:}\\
Seungwoo \textsc{Schin} \\% Your name


Hyunwook \textsc{Kang} %
\end{flushleft}
\end{minipage}
\begin{minipage}{0.4\textwidth}
\begin{flushright} \large
\emph{Typeset by:} \\
Seungwoo \textsc{Schin} % Supervisor's Name


Hyunwook \textsc{Kang} %
\end{flushright}
\end{minipage}\\[4cm]

% If you don't want a supervisor, uncomment the two lines below and remove the section above
%\Large \emph{Author:}\\
%John \textsc{Smith}\\[3cm] % Your name

\textsc{KAIST}\\[1.5cm] % Name of your university/college

%----------------------------------------------------------------------------------------
%	DATE SECTION
%----------------------------------------------------------------------------------------

%{\large \today}\\[3cm] % Date, change the \today to a set date if you want to be precise
2017 Fall Semester

%----------------------------------------------------------------------------------------
%	LOGO SECTION
%----------------------------------------------------------------------------------------

%\includegraphics{Logo}\\[1cm] % Include a department/university logo - this will require the graphicx package
 
%----------------------------------------------------------------------------------------

%\vfill % Fill the rest of the page with whitespace

\end{titlepage}

% Table of contents 

\tableofcontents
\newpage
\textsc{ Timeline \newline ~9/14 : HyunWook W1 + W2S1, Seungwoo W2S2 \newline  ~9/21 : HyunWook W3, Seungwoo W4}\\[1.5cm]
\newpage
%\section{강의노트 템플릿} 

본 문서는 \LaTeX 강의노트에 대한 것이며, pdflatex를 이용하여 윈도우 10에서 빌드되었다. 

\subsection{본 템플릿에 대한 간단한 소개}

\paragraph{문서 구조} 본 템플릿을 사용할 시, 크게 신경쓸 부분은 없을 것이다. 기본적으로는 html과 비슷한 구조이나, 환경 시작은 {\textbackslash}begin\{...\}로, 끝은 {\textbackslash}end\{...\}로 끝낸다. section이나 subsection, subsubsection은 굳이 열거나 닫지 않아도 된다. 명령어는 \textbackslash를 앞에 붙여서 나타낸다. 본 문서의 TeX 코드를 보면 대략적으로 어떤 식으로 쓰는지에 대해서 알 수 있으리라 생각된다. 각 강의를 들은 후 Summary.tex에 \textbackslash include \{파일이름\}으로 추가한 후 Summar.tex를 pdflatex로 두 번 컴파일하면 된다. 각주는 이렇게\footnote{footnote 명령어} 달 수 있다. 일반적인 \LaTeX 사용법에 대해서는 \href{http://legacy-wiki.dgoon.net/doku.php?id=latex:latex}{LaTeX wiki}나 \href{http://ftp.isu.edu.tw/pub/Unix/CTAN/info/lshort/korean/lshort-kr.pdf}{LaTeX 메뉴얼}을 참고하는 것을 추천하다.

\paragraph{소스 코드} 본 템플릿에서는 파이썬, 자바, C 소스 코드를 지원\footnote{다른 언어 지원이 필요하면 \href{mailto:principia\_12@kaist.ac.kr}{관리자} 에게 문의}한다. 

다시 수정 수정 
\begin{itemize} 

\item{Python} 

\begin{Python} 
def main:
    print('Hello, World!')
    
    return 0
    
\end{Python} 

\item{C}


\begin{C} 
#include <stdio.h>

int main(int argc, const char * argv[]) 
{
    printf("Hello, World!\n");
    return 0;
}    
\end{C} 

\item{Java}

\begin{Java} 
public class HelloWorld {
    public static void main(String[] args) {
        System.out.println("Hello, world!");
    }
}
\end{Java} 

\end{itemize}

    

\import{Deep Learning Specialization/}{C1W1_Introduction_to_deep_learning.tex}
\end{document}



























